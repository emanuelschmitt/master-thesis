\chapter{Data Preprocessing \label{cha:chapter4}}
\section{Overview}
1. Query 2. Split into Taps 3. Interpolate 4. Feature Extraction 5. CSV
\section{Feature Extraction}
- Explain column, matrix and sensor features.

\begin{figure}[h!]
  \centering
  \begin{minipage}{.3\textwidth}
    \centering
    \begin{tikzpicture}
      \matrix [matrix of math nodes] (m)
      {
          g_{x_0} &g_{x_1} &g_{x_2} &g_{x_3} &g_{x_4} \\
          g_{y_0} &g_{y_1} &g_{y_2} &g_{y_3} &g_{y_4} \\            
          g_{z_0} &g_{z_1} &g_{z_2} &g_{z_3} &g_{z_4} \\
          \_ & \_ & \_ & \_ & \_ \\      
          a_{x_0} &a_{x_1} &a_{x_2} &a_{x_3} &a_{x_4} \\
          a_{y_0} &a_{y_1} &a_{y_2} &a_{y_3} &a_{y_4} \\            
          a_{z_0} &a_{z_1} &a_{z_2} &a_{z_3} &a_{z_4} \\
      };
      \draw[color=black!100, thick] (m-1-1.north west) -- (m-1-5.north east) -- (m-1-5.south east) -- (m-1-1.south west) -- (m-1-1.north west);
      \draw[color=black!100, thick] (m-2-1.north west) -- (m-2-5.north east) -- (m-2-5.south east) -- (m-2-1.south west) -- (m-2-1.north west);
      \draw[color=black!100, thick] (m-3-1.north west) -- (m-3-5.north east) -- (m-3-5.south east) -- (m-3-1.south west) -- (m-3-1.north west);
      \draw[color=black!100, thick] (m-5-1.north west) -- (m-5-5.north east) -- (m-5-5.south east) -- (m-5-1.south west) -- (m-5-1.north west);
      \draw[color=black!100, thick] (m-6-1.north west) -- (m-6-5.north east) -- (m-6-5.south east) -- (m-6-1.south west) -- (m-6-1.north west);
      \draw[color=black!100, thick] (m-7-1.north west) -- (m-7-5.north east) -- (m-7-5.south east) -- (m-7-1.south west) -- (m-7-1.north west);
      \node[align = center, below=2cm]{Column Features};
    \end{tikzpicture}
  \end{minipage}
  \begin{minipage}{.3\textwidth}
    \centering
    \begin{tikzpicture}
      \matrix [matrix of math nodes] (m)
      {
          g_{x_0} &g_{x_1} &g_{x_2} &g_{x_3} &g_{x_4} \\
          g_{y_0} &g_{y_1} &g_{y_2} &g_{y_3} &g_{y_4} \\            
          g_{z_0} &g_{z_1} &g_{z_2} &g_{z_3} &g_{z_4} \\
          \_ & \_ & \_ & \_ & \_ \\      
          a_{x_0} &a_{x_1} &a_{x_2} &a_{x_3} &a_{x_4} \\
          a_{y_0} &a_{y_1} &a_{y_2} &a_{y_3} &a_{y_4} \\            
          a_{z_0} &a_{z_1} &a_{z_2} &a_{z_3} &a_{z_4} \\
      };
      \draw[color=black!100, thick] (m-1-1.north west) -- (m-1-5.north east) -- (m-3-5.south east) -- (m-3-1.south west) -- (m-1-1.north west);
      \draw[color=black!100, thick] (m-5-1.north west) -- (m-5-5.north east) -- (m-7-5.south east) -- (m-7-1.south west) -- (m-5-1.north west);      
      \node[align = center, below=2cm]{Sensor Features};     
      % \draw[color=black,8oublies-](m-1-2.north) -- +(0,0.3);
    \end{tikzpicture}
  \end{minipage}
  \begin{minipage}{.3\textwidth}
    \centering
    \begin{tikzpicture}
      \matrix [matrix of math nodes] (m)
      {
          g_{x_0} &g_{x_1} &g_{x_2} &g_{x_3} &g_{x_4} \\
          g_{y_0} &g_{y_1} &g_{y_2} &g_{y_3} &g_{y_4} \\            
          g_{z_0} &g_{z_1} &g_{z_2} &g_{z_3} &g_{z_4} \\
          \_ & \_ & \_ & \_ & \_ \\      
          a_{x_0} &a_{x_1} &a_{x_2} &a_{x_3} &a_{x_4} \\
          a_{y_0} &a_{y_1} &a_{y_2} &a_{y_3} &a_{y_4} \\            
          a_{z_0} &a_{z_1} &a_{z_2} &a_{z_3} &a_{z_4} \\
      };
      \draw[color=black!100, thick] (m-1-1.north west) -- (m-1-5.north east) -- (m-7-5.south east) -- (m-7-1.south west) -- (m-1-1.north west);
      % \draw[color=red,double,implies-](m-1-2.north) -- +(0,0.3);
      \node[align = center, below=2cm]{Matrix Feature};
    \end{tikzpicture}
  \end{minipage}
  \caption{The figure shows how different features are extracted from the overall sensor matrices: Column features, sensor features and matrix features.}
\end{figure}  

\subsection{Column Features}
\begin{center}
  % Feature Description
  \begin{tabular}{ | l| l | l | l |}
  \hline
  Name & Description & Feature Type & Amount \\
  \hline
    \texttt{peak} & Amount of peaks in the time series & column  & 6 \\
    \texttt{zero\_crossing} & Amount of zero crossings in the signal & column  & 6 \\ 
    \texttt{energy} & Energy of the signal & column  & 6 \\ 
    \texttt{entropy} & Entropy measure of the the signal & column  & 6 \\
    \texttt{mad} & Median absolute deviation of the signal & column  & 6 \\
    \texttt{ir} & Interquartile Range & column  & 6 \\
    \texttt{rms} & Root mean square of the signal & column  & 6 \\
    \texttt{mean} & Mean of the time series & column  & 6 \\
    \texttt{std} & Standard deviation of the time series & column  & 6 \\
    \texttt{min} & Minimum of the time series & column  & 6 \\
    \texttt{median} & Median of the time series & column  & 6 \\
    \texttt{max} & Maximal value of the time series & column  & 6 \\
    \texttt{var} & Variance of the time series & column  & 6 \\
    \texttt{skew} & Skewness of the time series & column  & 6 \\
    \texttt{kurtosis} & Kurtosis of the time series & column  & 6 \\
    \texttt{sem} & Standard error of the time series & column  & 6 \\
    \texttt{moment} & Moment in the time series & column & 6 \\
    \texttt{spline} & Spline interpolation of the signal (12 steps) & column  & 6*12 \\
  \hline
    \texttt{fro\_norm} & Frobenius matrix norm & matrix & 1 \\
    \texttt{inf\_norm} & Infinity matrix norm & matrix & 1 \\
    \texttt{l2\_norm} & L2 matrix norm & matrix & 1 \\
  \hline
    \texttt{cos\_angle} & Cosine Angle of sensor component pairs & sensor & 3 \\
    \texttt{pears\_cor} & Pearson Correlation of sensor component pairs & sensor & 3 \\
    
  \hline
  \end{tabular}
\end{center}

% peak,
% zero_crossing,
% energy,
% entropy,
% mad,
% iqr,
% rms,
% np.mean,
% np.std,
% np.min,
% np.median,
% np.max,
% np.var,
% st.skew,
% st.kurtosis,
% st.sem,
% st.moment
\subsection{Matrix Features}
\begin{center}
  % Feature Description
  \begin{tabular}{ | l | l |}
  \hline
  Name & Description \\
  \hline
    \texttt{fro\_norm} & - \\
    \texttt{inf\_norm} & - \\
    \texttt{l2\_norm} & - \\
  \hline
  \end{tabular}
\end{center}
\subsection{Sensor Features}
\section{Feature Scaling}
