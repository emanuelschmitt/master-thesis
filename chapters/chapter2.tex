\chapter{Fundamentals and Related Work\label{cha:chapter2}}

\section{User Interaction Inference}
% two words in interaction and 
Inferring user interactions through side channels has been of great interest to the academic world throughout history. As this thesis will cover a modern approach by recording sensory information provided by the Apple iPhone, an early and prominent example of spying on emanations reaches far back in time. \\

Back in 1943, researchers of the TEMPEST project, a subdivision of the NSA\footnote{National Security Agency}, were able to infer information from the infamous Bell Telephone model 131-B2, a teletype terminal which was used for encrypting wartime communication. Using an oscilloscope, researchers could capture leaking electromagnetic signals from the device and by carefully examining the peaks of the recorded signals, the plain message the device was currently processing could be reconstructed. This technique was later advanced and used in the the Vietnam war. Through similar electric emanation the US military could detect approaching Viet Cong trucks giving them an immense competitive advantage. \\

  % not having to modify the host system.

Ever since, various user interaction inference experiments have been conducted by researchers worldwide. However, in order to categorize these different approaches found in literature, we will divide these based on the emanation channel that was spied on. This categorization approach is more suitable since device model and user interaction strategies have frequently changed in time.\\ 

The literature denoted that user inference can be perused with the help of acoustic, optical, electro-magnetic and on sensory emanation. We will describe these in the following sections.

\subsection{Acoustic Emanation}
One frequently discussed method of obtaining leaked information involves the utilization of the acoustic channel. Many electronic devices deploy tiny mechanics that generate sounds as a byproduct during interactions or during operation. These distinct sounds can differ in their characteristics making them reconvertable to the original information currently being processed by the machine. In these scenarios the eavesdropper targets a microphone in near proximity of the target device in order to capture the audio signals the device is exposing on which learning algorithms are then applied. \\

In 2010, \citeauthor{Backes:2010:ASA:1929820.1929847} examined the problem of acoustic emanations of dot matrix printers, which where, at that time, still commonly used in banks and medical offices. By using a simple consumer-grade microphone, the researchers were able to recover whole sentences the printer was printing based on a record of sound. \citeauthor{Backes:2010:ASA:1929820.1929847} pre-processed the audio samples in order to extracted frequency-domain features feeding a hidden markov model, a technique commly used in audio speech recognition. As a result, the recognition system was able to reconstruct individual characters based on the sound inputs. To demonstrate a potential attack, the researchers deployed the system in a medical office where they were able to obtain up to 72\% of the sentences being printed on a medical subscription \cite{Backes:2010:ASA:1929820.1929847}. \\

Being inspired by the findings concerning the dot matrix printer, \citeauthor{1301311} investigated acoustic emanations produced by hitting keystrokes on a desktop and notebook keyboard. Following their hypothesis that each key has a macroscopic difference in the construction mechanics as well and a distinct reverberation caused by the position in the board, individual keystrokes were recorded \cite{1301311}. Researchers then extracted frequency domain features from the audio signals and passed them into a neural network. In an experiment performing 300 keystrokes, 79\% of the characters could be correctly recognized \cite{1301311}. As this technique required substantial training before recognition, other studies have reached similar accuracies using an unsupervised approach \cite{Zhuang:2009:KAE:1609956.1609959} on the one hand and by using acoustic dictionaries \cite{Berger:2006:DAU:1180405.1180436} on the other.

% Maybe some text to round things up...
\subsection{Optical Emanation}
Optical emanations and reflections also pose a potential source for information. Computer screens, for instance, mediate information through breams of light which are captured by our human eye. However, as these rays can reflect upon other surfaces they are vulnerable of being captured by an eavesdropper. \\

\citeauthor{1004358} developed a technique to eavesdrop on cathode-ray tube monitors at distance. The information displayed on the monitor can be reconstructed from its distorted or even diffusely reflected light by using easy to access compontents such as a photomultiplier tube and a fast enough computer to process the recordings \cite{1004358}. A similar approach that comprises reflections was shown by \citeauthor{4531151}, however focusing on LCD displays rather than CRT monitors. In this experiment, the researchers caught reflections in various objects that are commonly to be found in close proximity to a computer screen. Such objects included eyeglasses, tea pots, spoons and even plastic bottles. This work was later extended to additionally capture screens based on the reflections on the human eye's cornea \cite{5207653}.

\subsection{Electro-magnetic Emanation}

As electric currents flow through the through computer components, they emit electromagnetic waves to their near surrounding. These electromagnetic radiations can be picked up as side channels using sensitive equipment for the purpose of reconstructing these signals in form of data. 
% maybe move down a little
Side-band electromagnetic emanations are present in keyboards \cite{Vuagnoux:2009:CEE:1855768.1855769}, computer screens \cite{vanEck:1985:ERV:7307.7308,kuhn2004electromagnetic}, printers \cite{przesmycki2014measurement} and computer interfaces, such as USB 2 \cite{nowosielski2014compromising} and the parallel port \cite{serialcablearticle}.\\

%  It is possible in some
% cases to obtain information on the signals used inside
% the equipment when the radiation is picked up and the
% received signals are decoded. Especially in the case of
% digital equipment this possibility constitutes a
% problem, because remote reconstruction of signals
% inside the equipment may enable reconstruction of the
% data the equipment is processing
A prominent example of eavesdropping on electromagnetic emanations was detected by the researcher \citeauthor{vanEck:1985:ERV:7307.7308} discovering that cathode ray tube monitors could be spied upon from a distance using general market equipment \cite{vanEck:1985:ERV:7307.7308}. With the use of antennas \citeauthor{vanEck:1985:ERV:7307.7308} could receive the signals emitted from the monitor's cable. Since these cables only transmit the video signal for visualization, the researcher could display the visual output of the target monitor revealing a full screen cast of the original image. This attack is referred to in literature as \textit{Van Eck Phreaking}.\\

% These emissions lead to a full or a partial
% recovery of the keystrokes. We implemented these sidechannel
% attacks and our best practical attack fully recovered
% 95% of the keystrokes of a PS/2 keyboard at a distance
% up to 20 meters, even through walls. We tested
% 12 different keyboard models bought between 2001 and
% 2008 (PS/2, USB, wireless and laptop). They are all vulnerable
% to at least one of the four attacks. We conclude
% that most of modern computer keyboards generate compromising
% emanations (mainly because of the manufacturer
% cost pressures in the design). Hence, they are not
% safe to transmit confidential information.

\cite{Vuagnoux:2009:CEE:1855768.1855769} used electromagnetic emanations of wireless and cable keyboard to recover keystrokes from a distance \cite{}. 

- Smart Cards \cite{Quisquater:2001:EAM:646803.705980}\\
- Wireless keyboards \cite{Vuagnoux:2009:CEE:1855768.1855769}\\
- Serial port cables \cite{serialcablearticle} \\
- CMOS \cite{Agrawal2003} \\
- CRT radiation \cite{vanEck:1985:ERV:7307.7308}

\subsection{Motion Emanation}
In the past decade modern devices are increasingly equipeed with highly responsive sensors, such as the gyroscope and accelerometer enabling the devices to sense rich interactions with their environment. As user interactions, such as typing the keyboard or tapping on touchscreens, require the user to apply a certain force while entering information, this motion can be captured by motion sensors in order to be used for as a side-channel attack reconstructing secret information \cite{Tapprints,Accessory,Touchlogger}. \\

\citeauthor{Marquardt:2011:IDV:2046707.2046771} conducted an experiment where an Apple iPhone equpped with an application that captures motion with it's accelerometer was placed next to a keyboard. Subjects then had to enter sentences on the keyboard while the application was monitoring the user's motion while typing. Furthermore, The researchers could then decode the accelerometer signals by measuring the relative physical position and distance between each vibration. The decoded characters were then matched based on a dictionary containing a frequency distribution of commonly used word. As a result, words were successfully obtained with an accuracy of up to 80\% \cite{Marquardt:2011:IDV:2046707.2046771}.


\section{Eavesdropping on touch screen user interactions}
As we have seen in the previous section, user interactions with peripheral devices, such as the keyboard or PIN pads, can be obtained by either the acoustic channel, by electromagnetical leakage or by capturing the motion of the user. However, with the rise in smartphone and tablet computer usage, the same discussed methods that work for a keyboard do not apply to tap interactions on a surface screen. Since a touchscreen does not embody fine mechanics producing sounds nor does it not seem to emanate electromagnetic radiations, the research community has revealed nouvelle methods to spy on user taps based on the smartphone embodied motion sensors.\\

The general idea behind the three approaches that are going to be discuss in the following is that a tap, or to be more precise the magnitude of the force of the tap, on a specific touch screen location creates an identifiable pattern on the motion sensors that can be sufficient to infer the initial tap location. This is particularly interesting since motion sensors are not considered as being private-sensitive and therefore lack access restrictions by the operating systems of the devices.\\

The first paper regarding this security threat was published by \citeauthor{Touchlogger}. In their proof-of-concept study they created an Android\footnote{Android operating system for smartphones by Google Inc.} application which displays a 10-digit PIN-pad. During interactions with the PIN-pad, the accelerometer signals were monitored and used for later data analysis. Having observed that a tap movement affects the rotation angle of the screen, the researchers have handcrafted features based on the path of the \textit{pitch}\footnote{The pitch-angle corresponds to the x-axis of the accelerometer.} and the \textit{roll}\footnote{The roll angle corresponds to the y-axis of the accelerometer.} angles of the accelerometer to find a dominating edge on where the tap has originally taken place. By using a probability density function for a Gaussian distribution the researchers were able to reach an accuracy of 70\% for interred PIN-pad digits on a training-set size of 449 pin strokes \cite{Touchlogger}. However, as \textit{Touchlogger} was a promising first step, due to it's low granularity of only 10 distinguishable large screen areas, it remained unclear if the attack can be carried over to a full software keyboard. Furthermore, since the inference was performed on only a single smartphone model, it leaves the question open if other smartphones or tablet computers were similarly vulnerable.\\

In order to show the feasibility for a full software keyboard, \citeauthor{Accessory} created a second attempt to the Problem by creating \textit{ACCessory}. \textit{ACCessory} is an Android application with functionalities similar to the previously mentioned \textit{Touchlogger}. However, the application differ in it's granularity by providing two modes for tap inputs. The first consists of tap areas arranged in a 60-cell grid whereas the second is a QWERTY keyboard in landscape orientation. Having extracted features mainly from the time-domain, a classification using the Random Forests algorithm reached an accuracy of 24.5\% for the 60-cell grid with dataset containing . The experiment on the software keyboard was aimed at cracking passwords, on which 

Touchlogger\\
- first paper\\
- Proof of concept
- Granularity: Pin pad 10 digits - 70 accuracy\\
- probability density function
- Android HTC - EVO\\

Accessory\\
- Android 
- Areamode 60 grid + qwerty keyboard
- Random Forest classifier + SVM + MLP 
- Analysis of text passwords
- Accuracy: 25\% on grid and 
- 1300 keystrokes grid + 2700 on character interference




- \cite{Touchlogger} \\
- \cite{Tapprints} \\
- \cite{Accessory} \\

\newpage
\section{Machine Learning \label{sec:fieldstudy}}
As we will be using machine learning techniques for the later classification of sensory data, the following chapter will give a brief overview of the fundamental concepts evolving around statistical learning.

\subsection{Overview and Definition}

Ever since computers were invented, there has been a desire to enable them to learn \cite{samuel2000some}. This desire has grown into the field of machine learning which seeks to answer questions on how to build build systems that automatically improve with experience, and what the fundamental laws of learning processes are. Today, state-of-the-art ML covers a large set of methods and algorithms designed to accomplish tasks where conventional hard-coded routines have brought insufficient results. From speech recognition to email spam detection or recommendation systems, ML methods find broad usage in a variety of problem domains. \\

In order to understand what the principle of machine learning is, we will start with a definition by Samuel \cite{samuel2000some}:\\
\begin{quote}
\textit{Machine learning is the field of study that gives computers the ability to learn without being explicitly programmed.}\\
\end{quote}

In this definition, special emphasis is to be put on the last part of this definition. A computer is only then able to learn when he can perform a task without being explicitly instructed. Thus, in order to learn, the computer must somehow be able to instruct itself without the influence of an outer . As this definition lacks a more detailed view on what computer learning is, we will dive into a definition by Tom Mitchell \cite{mitchell2006discipline}:\\
\begin{quote}
\textit{A learning system is said to learn from experience E with respect to some class of tasks T and performance measure P, if its performance at tasks in T, as measured by P, improves with experience E.}\\
\end{quote}

The example that Mitchell notes, is one from the games of checkers \cite{mitchell2006discipline}. In this case checkers is the task T that the computer is aiming to learn. In order for the computer to learn, information on previously played matches is required. Since the computer does not know how to evaluate is a particular match was either good or bad, we set the performance to be defined based on how many matches were actually won. If a computer program can raise the amount of games won \textit{(performance measure P)} with the help of the experience from previously placed matches \textit{(experience E)} then it can learn to play checkers \textit{(task T)}. \\

% TODO: Maybe delete this.
To break this down into a more practical perspective, the challenge lies in finding an appropriate model in order to learn from data, which is the most common format to represent past experience. By learning, the computer adjusts parameters on the model based on the data that we feed the system with. Once the model has been adjusted, it can perform tasks with new incoming experience.

\subsection{Categorization of Methods}
As machine learning algorithms and methods differ from their approach to learning and underlying concepts, it is common practice to separate these into the following categories \cite{Duda:2000:PC:954544, Marsland:2009:MLA:1571643}  : Supervised learning, unsupervised learning, reinforcement learning and evolutionary learning. In the following sections I will briefly outline these.\\

\begin{itemize}

  \item[] \textbf{Supervised learning}, which is also named learning from example, is presumably the most prominent category of ML algorithms. The algorithm is given a training set of examples $\{x_0, \dots, x_n \}$, which are also known as \textit{features} and the correct target values $\{y_0, \dots, y_n \}$ mapped to each set of features, which is the answer that the algorithm should produce. The algorithm then generalizes based on the training set in order to respond with sensible outputs on all possible input values. Outputs, if they are discrete labels, correspond to a classification task whereas outputs on a continuous scale refer to a regression task (see \cite{Marsland:2009:MLA:1571643}).

An example for supervised learning is the classification of malignant or benign tumors as seen in cancer diagnosis. Let's assume we have a dataset with different properties of a tumor, such as the size or the color of the cells. These properties form our features $x$. Each set of features is mapped to an output label $y$ stating if the tumor is malignant or benign. The first step is to use the pairs $(x, y)$ of the training set to teach the algorithm the correct mapping of the problem space. As $x$ is linked to the output $y$ in the training set, learning the conjunction of these two values is done under supervision since the output label $y$ is given. Once learned, the algorithm is generalized to map unseen inputs to the correct output label.

Practical applications are for example digit and handwriting recognition \cite{lecun1990handwritten}, spam filtering \cite{guzella2009review} for e-mails or network anomaly detection \cite{lee2010uncovering}.

Presumably the most widely known machine learning techniques belong to this category, such as Support Vector Machines (SVMs), Artificial Neural Networks, Bayesian Statistics, Random Forests and Decision Trees \cite{Duda:2000:PC:954544}.\\

%TODO: add referation to how we will solve issues in the thesis.

  \item[] \textbf{Unsupervised learning} is the task of learning structures from input values that are not explicitly labeled. In comparison to supervised learning, where correct output values are provided for each input, unsupervised algorithms learn to identify similarities in the input data and can therefore group these \cite{Duda:2000:PC:954544}. These grouping problems are referred to as \textit{clustering}. The underlying idea here, is that humans learn by not explicitly being told what the right answer should be \cite{Marsland:2009:MLA:1571643}. If a human sees different species of snakes, for instance, he or she is able to identify them all as snakes. Hence, the human is aware that there are differences in each specific type of snake without specifically knowing a correct label.

A prominent example where unsupervised learning is heavily used, is in recommender systems for online retail shops. Amazon.com, for instance, uses a technique called \textit{collaborative filtering}, which measures similarity in customers based that they have previously bought \cite{linden2003amazon}. Having identified similar customers utilizing the cosine similarity, the algorithm can then recommend items that similar users have bought. This technique is also used for music recommendations \cite{perez2017recommender} or social network recommendations \cite{kautz1997referral}.

The field of unsupervised learning is closely related to density estimation in statistics, as with the density of inputs, we are able to group them. The K-means algorithm is the most prominent in this field \cite{Marsland:2009:MLA:1571643}.\\
  
  \item[] \textbf{Reinforcement learning} falls in between supervised and unsupervised learning methods. Whereas supervised learning tries to bridge the gap between input and corresponding output values and unsupervised methods detect groupings in incoming data, reinforcement learning is based on learning with a \textit{critic} \cite{Marsland:2009:MLA:1571643}. The algorithm tries different solution strategies to a problem and is told weather or not the answer provided was correct. An important fact here, is that the algorithm is not told how to correct itself. This practice of "trying-out" is based on the concept of \textit{trail-and-error learning} which is known as the \textit{Law-of-effect} \cite{Marsland:2009:MLA:1571643}. A good example is a child that tries to stand up and learn walking. The child tries out many different strategies for staying upright and receives feedback from the field based on how long it can stand without falling down again. The method that previously worked best is then repeated in order to find the optimal solution resulting in the child learning to walk \cite{Marsland:2009:MLA:1571643}.
  
 In more mathematical terms, the reinforcement learning problem is formalized with an agent and his environment. The environment in which the agent is set provides a set of \textit{states} on which the agent can perform \textit{actions} to maximize a certain \textit{reward}. By performing actions the state changes and a new reward is calculated. The reward then tells the agent if the action was a good choice. Goal of the algorithm is to maximize the reward \cite{Marsland:2009:MLA:1571643}.
 
 Reinforcement learning is a practical computational tool for constructing autonomous systems that improve themselves with experience. These applications have ranged from robotics, to industrial manufacturing, to combinatorial search problems such
as computer game playing \cite{kaelbling1996reinforcement}. 
Prominent methods of this category are Q-learning, Monte Carlo
methods and Hidden Markov Models \cite{Marsland:2009:MLA:1571643}.\\

  \item[] \textbf{Evolutionary learning} is inspired by strongly inspired by nature. As biological evolution improves the survival of a species, the strategy of adaptation to improve survival rates and the chance of offspring has inspired researchers to craft genetic algorithms (GA) \cite{Marsland:2009:MLA:1571643}. 

Genetic algorithms are a family of adaptive search procedures which have derived their name from the fact that they are based on models of genetic change in a population of individuals. These models have their foundation in three basic ideas: (1) Each evolutionary state of a population can be evaluated on a \textit{fitness} scale. This is done since biological evolution has a natural bias towards animals that are \" fitter \" than others. These animals tend to live longer, are more attractive and generate healthier and happier offspring, an idea which was originated in Charles Darwin's \"The Origin of Species\". (2) Each population can be mated to generate offspring using a \textit{mating operator}. (3) The third component are \textit{genetic operator}, such as \textit{crossover} and \textit{mutation}, which determine how the offspring solution is composed of the genetic material of the parents \cite{de1988learning}.

Evolutionary learning is often considered when other methods fail to find a reasonable answer. Algorithms find applications in search and mathematical optimization, but also in arts and simulation \cite{Marsland:2009:MLA:1571643}.\\
\end{itemize}

In this section we have seen several different problems that we can solve with the help of algorithmic learning. For our use case, as we want to predict the locations on smartphone screens using sensory data. As this is a supervised learning problem, we will cover one supervised approach in more detail in the following section: Artificial neural networks.

\subsection{Artificial Neural Networks}
% Add this if we consider using neural Networks...

\subsection{Regularization}
\subsection{Optimization}


\section{Data Aquisition in the field \label{sec:fieldstudy}}
- topics on data aquisition in the field