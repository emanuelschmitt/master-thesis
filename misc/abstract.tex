\thispagestyle{empty}
\vspace*{1.0cm}

\begin{center}
    \textbf{Abstract}
\end{center}

\vspace*{0.5cm}

\noindent

Research has shown that the location of touchscreen taps on modern smartphones and tablet computers can be identified based on sensor recordings from the device's accelerometer and gyroscope readings. This security threat implies that an attacker could launch a background process on the mobile device and send the motion sensor readings to a third party vendor for further analysis. Even though the inference is a non-trivial task requiring machine learning algorithms in order to predict the tap location, yet previous research has shown that PINs and passwords of users could be obtained. However, as the tap location inference was only shown for taps generated in a controlled setting not reflecting the environment users naturally engage with their smartphones, this work bridges this gap.

In this work, I propose TapSensing, a data acquisition system designed to collects touchscreen tap event information with corresponding accelerometer and gyroscope readings. Having performing a data acquisition study with 27 subjects and 3 different iPhone models, a total of 45,000 labeled taps could be acquired from a laboratory and field environment enabling a direct comparison of both environments. The overall findings show that tap location inference is generally possible for data acquired in the field, however, with a performance reduction of approximately 20\% when comparing both environments. As the tap inference has therefore been shown for a more realistic data set and by aligning with the previous experiments, this work yet again confirms that smartphone motion sensors could potentially be used to comprise the user's privacy.