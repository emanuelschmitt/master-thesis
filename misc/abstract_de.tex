\thispagestyle{empty}
\vspace*{0.2cm}

\begin{center}
    \textbf{Zusammenfassung}
\end{center}

\vspace*{0.5cm}

\noindent

Untersuchungen haben gezeigt, das die Position eines Taps auf einem Smartphone oder Tablet Displays \"uber eine Aufzeichnung der Lagesensoren bestimmt werden kann. Diese Sicherheitsl\"ucke deutet darauf hin, dass ein potentieller Angreifer einen Hintergrundprozess auf dem Zielger\"at installieren und die aufgezeichneten Daten an eine serverseitige Applikation zur weiteren Analyse schicken k\"onnte. Obwohl die Bestimmung der Postion auf der Touchscreen keine einfache Aufgabe darstellt, die nur mit Hilfe von Maschi-nellem Lernen m\"oglich ist, ist es Forschern dennoch gelungen PINs und Passw\"orter von einzelnen Nutzern auszusp\"ahen. Da jedoch die Daten zur Vorhersage in den bisherigen Versuchen von Nutzern stammen, die in einem Laborversuch aufgezeichnet wurden und die Daten hierbei keine reale Nutzerumgebung darstellen, schliesst diese Arbeit diese L\"ucke.

In dieser Arbeit stelle ich TapSensing vor, eine Applikation die Touch Screen Tap events mit den entsprechenden Lagesensor-Daten aufzeichnet. In einem Labor sowie in einem Feldversuch mit insgesamt 27 Teilnehmer, wurden 45.000 einzelne Taps aufgenommen. In einer Analyse konnte gezeigt werden, dass eine Vorhersage der Tap-Position auf dem Display ebenso f\"ur Daten aus dem Feld m\"oglich ist. Im direkten Vergleich zwischen Feld und Labor wurden jedoch Performance-Einbußen von rund 20\% f\"ur Daten aus dem Feld gemessen. Da die Tap Positions Inferenz nun f\"ur reale Konditionen gezeigt werden konnte, k\"onnen wir uns bisherigen Untersuchungen anschließen und erneut best\"tigen, dass Lagesensoren ein erhebliches Sicherheitsrisiko f\"ur den Nutzer darstellen.
% Die Forschung hat gezeigt, dass der Standort von Touchscreen-H\"ahnen auf modernen Smartphones und Tablet-Computern basierend auf Sensoraufzeichnungen von Beschleunigungsmesser- und Gyroskop-Messwerten des Ger\"ats identifiziert werden kann. Diese Sicherheitsbedrohung impliziert, dass ein Angreifer einen Hintergrundprozess auf dem Mobilger\"at starten und die Bewegungssensordaten zur weiteren Analyse an einen Drittanbieter senden kann. Obwohl dies eine nicht triviale Aufgabe ist, die Algorithmen zum maschinellen Lernen erfordert, um den Abzweigort vorherzusagen, haben bisherige Untersuchungen gezeigt, dass PINs und Kennw\"orter von Benutzern erhalten werden k\"onnen. Da die Klopfort-Inferenz jedoch nur f\"ur Taps dargestellt wurde, die in einer kontrollierten Umgebung erzeugt wurden, die die Umgebung nicht widerspiegelt, die Benutzer von Natur aus mit ihren Smartphones besch\"aftigen, versucht diese Arbeit diese L\"ucke zu schließen.

% In dieser Arbeit empfehle ich TapSensing, ein Datenerfassungssystem, das Touchscreen-Tap-Event-Informationen mit entsprechenden Beschleunigungsmessern und Gyroskop-Messwerten erfasst. Mit einer Datenakquisitionsstudie mit 27 Probanden und 3 verschiedenen iPhone-Modellen konnten insgesamt 45.000 markierte Abgriffe aus einer Labor- und Feldumgebung gewonnen werden, die einen Vergleich erm\"oglichen. Die Gesamtergebnisse zeigen, dass eine Klopfortinferenz generell f\"ur im Feld erfasste Daten m\"oglich ist, jedoch mit einer Leistungsreduzierung von ungef\"ahr 20%, wenn beide Umgebungen verglichen werden. Da die Tipping-Inferenz daher f\"ur einen realistischeren Datensatz gezeigt wurde, und indem sie sich an die vorherigen Experimente anlehnt, best\"atigt diese Arbeit erneut, dass Smartphone-Bewegungssensoren potentiell verwendet werden k\"onnen, um die Privatsph\"are des Benutzers zu umfassen.


\noindent 